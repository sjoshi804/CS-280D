%=======================02-713 LaTeX template, following the 15-210 template==================
%
% You don't need to use LaTeX or this template, but you must turn your homework in as
% a typeset PDF somehow.
%
% How to use:
%    1. Update your information in section "A" below
%    2. Write your answers in section "B" below. Precede answers for all 
%       parts of a question with the command "\question{n}{desc}" where n is
%       the question number and "desc" is a short, one-line description of 
%       the problem. There is no need to restate the problem.
%    3. If a question has multiple parts, precede the answer to part x with the
%       command "\part{x}".
%    4. If a problem asks you to design an algorithm, use the commands
%       \algorithm, \correctness, \runtime to precede your discussion of the 
%       description of the algorithm, its correctness, and its running time, respectively.
%    5. You can include graphics by using the command \includegraphics{FILENAME}
%
\documentclass[11pt]{article}
\usepackage{amsmath,amssymb,amsthm}
\usepackage[noend]{algpseudocode}
\usepackage{graphicx}
\usepackage[margin=1in]{geometry}
\usepackage{mathtools}
\usepackage{fancyhdr}
\usepackage{algorithmicx}
\setlength{\parindent}{0pt}
\setlength{\parskip}{5pt plus 1pt}
\setlength{\headheight}{13.6pt}
\newcommand\question[2]{\vspace{.25in}\hrule\textbf{#1: #2}\vspace{.5em}\hrule\vspace{.10in}}
\renewcommand\part[1]{\vspace{.10in}\textbf{(#1)}}
\newcommand\analysis{\vspace{.10in}\textbf{Analysis: }\newline}
\newcommand\algorithm{\vspace{.10in}\textbf{Algorithm: }}
\newcommand\correctness{\vspace{.10in}\textbf{Correctness: }\newline}
\newcommand\runtime{\vspace{.10in}\textbf{Running time: }\newline}
\pagestyle{fancyplain}
\lhead{\textbf{\NAME}}
\chead{\textbf{CS 280D HW\HWNUM}}
\rhead{\today}
\begin{document}\raggedright
%Section A==============Change the values below to match your information==================
\newcommand\NAME{Siddharth Joshi}  % your name
\newcommand\HWNUM{4}              % the homework number
%Section B==============Put your answers to the questions below here=======================

% no need to restate the problem --- the graders know which problem is which,
% but replacing "The First Problem" with a short phrase will help you remember
% which problem this is when you read over your homeworks to study.

\question{1}{Adapting Byzantine Agreement Algorithm in Synchronous Message Passing for Omission Faults with Polynomial Message Complexity} 

\analysis
The problem of solving consensus when only omission faults can occurs is essentially guaranteeing that all processors exclude the same processors in case of an asymmetric omission error where some processors received a message but others didn't. In particular, this is solvable in a manner identical to the solution for byzantine agreement in a synchronous setting by simply ignoring the messages from processors that have exhibited omission errors. 
A processor can identify the processors exhibiting omission errors either if it itself doesn't receive a message from a particular processor in a certain round or if another processor informs it in its message that it didn't receive a message from said processor. 

\algorithm
\begin{algorithmic}
\State Send your id to everyone \algorithmiccomment{Round 1}
\State Let S = the set of all processor's ids (include your own) \algorithmiccomment{No message received represented by $\bot$}
\State Send S to everyone \algorithmiccomment{Round 2}
\State Let $S_i$ = the set of ids received from $P_i$ $\forall i$ 
\While {S and all $S_i$s are not equal}
    \For {every processor $P_j$}
        \If {corresponding value in S or any of the sets $S_i$ = $\bot$ or no message received from $P_j$ in this round}
            \State Set value corresponding to $P_j$ in S to $\bot$
        \EndIf
    \EndFor
    \State Send S to everyone \algorithmiccomment{Round k for some k $\leq$ t}
\EndWhile
\State Output majority of S
\end{algorithmic}

\runtime
Since the system has t faults there must be around in t + 1 rounds when no new faults occur (by pigeon hole principle) and in this round the termination condition will be satisfied. Therefore this algorithm will terminate in t + 1 rounds.

\correctness
The algorithm is correct as since there is a round where no new faults occur there will be a point where all the received sets $S_i$ s and S are equal. But then outputing the majority of S for every processor will have the same result as the local set S is identical across all processors. This ensures that the criteria of consensus will be met (triviality due to majority function and validity due to all having same majority). Moreover, this algorithm will have polynomial message complexity as in every round every processor sends a set containing n messages (assuming there is a message for itself as well) and all n processors do so in every round therefore, message complexity = O($n^2$) = poly(n)
\end{document}










