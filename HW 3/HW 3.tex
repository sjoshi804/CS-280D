%=======================02-713 LaTeX template, following the 15-210 template==================
%
% You don't need to use LaTeX or this template, but you must turn your homework in as
% a typeset PDF somehow.
%
% How to use:
%    1. Update your information in section "A" below
%    2. Write your answers in section "B" below. Precede answers for all 
%       parts of a question with the command "\question{n}{desc}" where n is
%       the question number and "desc" is a short, one-line description of 
%       the problem. There is no need to restate the problem.
%    3. If a question has multiple parts, precede the answer to part x with the
%       command "\part{x}".
%    4. If a problem asks you to design an algorithm, use the commands
%       \algorithm, \correctness, \runtime to precede your discussion of the 
%       description of the algorithm, its correctness, and its running time, respectively.
%    5. You can include graphics by using the command \includegraphics{FILENAME}
%
\documentclass[11pt]{article}
\usepackage{amsmath,amssymb,amsthm}
\usepackage[noend]{algpseudocode}
\usepackage{graphicx}
\usepackage[margin=1in]{geometry}
\usepackage{mathtools}
\usepackage{fancyhdr}
\usepackage{algorithmicx}
\setlength{\parindent}{0pt}
\setlength{\parskip}{5pt plus 1pt}
\setlength{\headheight}{13.6pt}
\newcommand\question[2]{\vspace{.25in}\hrule\textbf{#1: #2}\vspace{.5em}\hrule\vspace{.10in}}
\renewcommand\part[1]{\vspace{.10in}\textbf{(#1)}}
\newcommand\analysis{\vspace{.10in}\textbf{Analysis: }\newline}
\newcommand\algorithm{\vspace{.10in}\textbf{Algorithm: }}
\newcommand\correctness{\vspace{.10in}\textbf{Correctness: }\newline}
\newcommand\runtime{\vspace{.10in}\textbf{Running time: }\newline}
\pagestyle{fancyplain}
\lhead{\textbf{\NAME}}
\chead{\textbf{CS 280D HW\HWNUM}}
\rhead{\today}
\begin{document}\raggedright
%Section A==============Change the values below to match your information==================
\newcommand\NAME{Siddharth Joshi}  % your name
\newcommand\HWNUM{}              % the homework number
%Section B==============Put your answers to the questions below here=======================

% no need to restate the problem --- the graders know which problem is which,
% but replacing "The First Problem" with a short phrase will help you remember
% which problem this is when you read over your homeworks to study.

\question{1}{Proving my HW 2 solution} 
\analysis
The main idea is that for the reported value of any processor's bit - a majority consensus is easy to achieve regardless of whether it has its communication compromised or not. Once this is known, the algorithm must just guarantee triviality as well and it will be complete. This can be done by having multiple such consensuses and a consensus of these consensuses of some sorts. The algorithm presented below articulates this formally. 

\algorithm
\begin{algorithmic}
\State Send your id to all processors \algorithmiccomment{Round 1}
\State Send all the ids you received to all processors \algorithmiccomment{Round 2}
\State Let $x_0$ = majority of the 3 bits that the other processors claimed was the bit of $P_0$
\State Let $x_1$ = majority of the 3 bits that the other processors claimed was the bit of $P_1$
\State Let $x_2$ = majority of the 3 bits that the other processors claimed was the bit of $P_2$
\State Output the majority of $x_0, x_1 and x_2$
\end{algorithmic}

\runtime
This algorithm terminates with an output in 2 rounds.

\correctness
Henceforth the term byzantine processor is used to refer to the processor that is affected by the adversary and suffers a byzantine fault. \newline

Assume $P_0$ is the byzantine processor, it can send any bit it wants to the other 3 processors but two of them must receive the same bit as a bit can take only two values (Pigeon-hole principle)
Therefore, $x_0$ for $P_0$ = this bit (the one received by 2 processors) as it will get back the 3 messages it sent for its own bit but also since $P_0$ is the byzantine processor, $P_1, P_2 and P_3$ must relay information perfectly, they too must have the same 3 messages. (This is because of the passing of received bits in round 2).\newline
Assume now $P_0$ is not the byzantine processor, $P_1, P_2 and P_3$ must receive the same bit but one of them may not convey this value accurately to another, however since $P_0$ and the other processor will faultlessly convey the correct bit the adversary's sabotage will not affect any processor from obtaining the correct bit.
\newline
Hence all processors will always agree on $x_0$ but by symmetry, it is easy to verify that this holds true in a similar manner for $x_1$ and $x_2$. 
\newline
Therefore now that it is shown that the processors will come to a consensus, all that remains to show is that this algorithm doesn't break the trivial condition as well i.e. some processor must have had the output bit as its input. Note here using the previous explanation that when $P_0$ is not the byzantine processor, $x_0$ = the true bit of $P_0$. Since only one processor can be byzantine, at least 2/3 of $x_0, x_1 and x_2$ must be the true bit of their respective processors. There are now two cases: these 2 bits are the same in which case this will be the majority and thus the output for all processors - triviality is not broken; or they aren't the same in which case triviality is maintained regardless of which bit is output. 
\newline

To conclude, since this algorithm shows that the processors come to a consensus while not breaking the trivial condition, it is a correct algorithm. 
\newpage

\question{2}{t+1 election in a t-resilient byzantine asynchronous system} 

\analysis
It is interesting to note that processors do know from whom they receive a message therefore an adversary capable of creating Byzantine faults cannot have a processor \lq pretend \rq to be another processor. As a result even in the byzantine asynchronous model in the first round, the adversary can at most delete a message, any incorrect behavior in this round must be a deletion as it is meaningless to claim to be any other processor as the receiving processor would be able to trivially verify this (as it knows who is sending a message). Therefore in the first round the asynchronous byzantine model reduces to the asynchronous omission model and thus admits a solution to t+1 election t-resiliently in the trivial manner - by simply using the solution for t+1 election from the omission model (as it outputs in 1 round, each processor using the simplex to accurately know which value it must output). 
\end{document}










